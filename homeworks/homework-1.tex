\documentclass{article}

% Packages required to support encoding
\usepackage{ucs}
\usepackage[utf8x]{inputenc}

% Packages required by code


% Packages always used
\usepackage{hyperref}
\usepackage{xspace}
\usepackage[usenames,dvipsnames]{color}
\hypersetup{colorlinks=true,urlcolor=blue}

\usepackage{geometry}
\geometry{letterpaper,textwidth=350pt,textheight=680pt,tmargin=60pt,
            left=72pt,footskip=24pt,headsep=18pt,headheight=14pt}
\usepackage{amsmath}
\usepackage{amssymb}
\usepackage{textcase}
\usepackage{soul}

\newcommand{\mat}[1]{\boldsymbol{#1}}
\renewcommand{\vec}[1]{\boldsymbol{\mathrm{#1}}}
\newcommand{\vecalt}[1]{\boldsymbol{#1}}

\newcommand{\conj}[1]{\overline{#1}}

\newcommand{\normof}[1]{\|#1\|}
\newcommand{\onormof}[2]{\|#1\|_{#2}}

\newcommand{\itr}[2]{#1^{(#2)}}
\newcommand{\itn}[1]{^{(#1)}}

\newcommand{\eps}{\varepsilon}
\newcommand{\kron}{\otimes}
\DeclareMathOperator{\diag}{diag}
\DeclareMathOperator{\trace}{trace}

\newcommand{\prob}{\mathbb{P}}
\newcommand{\probof}[1]{\prob\left\{ #1 \right\}}

\newcommand{\pmat}[1]{\begin{pmatrix} #1 \end{pmatrix}}
\newcommand{\bmat}[1]{\begin{bmatrix} #1 \end{bmatrix}}
\newcommand{\spmat}[1]{\left(\begin{smallmatrix} #1 \end{smallmatrix}\right)}
\newcommand{\sbmat}[1]{\left[\begin{smallmatrix} #1 \end{smallmatrix}\right]}

\newcommand{\RR}{\mathbb{R}}
\newcommand{\CC}{\mathbb{C}}

\providecommand{\eye}{\mat{I}}
\providecommand{\mA}{\ensuremath{\mat{A}}}
\providecommand{\mB}{\ensuremath{\mat{B}}}
\providecommand{\mC}{\ensuremath{\mat{C}}}
\providecommand{\mD}{\ensuremath{\mat{D}}}
\providecommand{\mE}{\ensuremath{\mat{E}}}
\providecommand{\mF}{\ensuremath{\mat{F}}}
\providecommand{\mG}{\ensuremath{\mat{G}}}
\providecommand{\mH}{\ensuremath{\mat{H}}}
\providecommand{\mI}{\ensuremath{\mat{I}}}
\providecommand{\mJ}{\ensuremath{\mat{J}}}
\providecommand{\mK}{\ensuremath{\mat{K}}}
\providecommand{\mL}{\ensuremath{\mat{L}}}
\providecommand{\mM}{\ensuremath{\mat{M}}}
\providecommand{\mN}{\ensuremath{\mat{N}}}
\providecommand{\mO}{\ensuremath{\mat{O}}}
\providecommand{\mP}{\ensuremath{\mat{P}}}
\providecommand{\mQ}{\ensuremath{\mat{Q}}}
\providecommand{\mR}{\ensuremath{\mat{R}}}
\providecommand{\mS}{\ensuremath{\mat{S}}}
\providecommand{\mT}{\ensuremath{\mat{T}}}
\providecommand{\mU}{\ensuremath{\mat{U}}}
\providecommand{\mV}{\ensuremath{\mat{V}}}
\providecommand{\mW}{\ensuremath{\mat{W}}}
\providecommand{\mX}{\ensuremath{\mat{X}}}
\providecommand{\mY}{\ensuremath{\mat{Y}}}
\providecommand{\mZ}{\ensuremath{\mat{Z}}}
\providecommand{\mLambda}{\ensuremath{\mat{\Lambda}}}
\providecommand{\mPbar}{\bar{\mP}}

\providecommand{\ones}{\vec{e}}
\providecommand{\va}{\ensuremath{\vec{a}}}
\providecommand{\vb}{\ensuremath{\vec{b}}}
\providecommand{\vc}{\ensuremath{\vec{c}}}
\providecommand{\vd}{\ensuremath{\vec{d}}}
\providecommand{\ve}{\ensuremath{\vec{e}}}
\providecommand{\vf}{\ensuremath{\vec{f}}}
\providecommand{\vg}{\ensuremath{\vec{g}}}
\providecommand{\vh}{\ensuremath{\vec{h}}}
\providecommand{\vi}{\ensuremath{\vec{i}}}
\providecommand{\vj}{\ensuremath{\vec{j}}}
\providecommand{\vk}{\ensuremath{\vec{k}}}
\providecommand{\vl}{\ensuremath{\vec{l}}}
\providecommand{\vm}{\ensuremath{\vec{l}}}
\providecommand{\vn}{\ensuremath{\vec{n}}}
\providecommand{\vo}{\ensuremath{\vec{o}}}
\providecommand{\vp}{\ensuremath{\vec{p}}}
\providecommand{\vq}{\ensuremath{\vec{q}}}
\providecommand{\vr}{\ensuremath{\vec{r}}}
\providecommand{\vs}{\ensuremath{\vec{s}}}
\providecommand{\vt}{\ensuremath{\vec{t}}}
\providecommand{\vu}{\ensuremath{\vec{u}}}
\providecommand{\vv}{\ensuremath{\vec{v}}}
\providecommand{\vw}{\ensuremath{\vec{w}}}
\providecommand{\vx}{\ensuremath{\vec{x}}}
\providecommand{\vy}{\ensuremath{\vec{y}}}
\providecommand{\vz}{\ensuremath{\vec{z}}}
\providecommand{\vpi}{\ensuremath{\vecalt{\pi}}}

\sodef\allcapsspacing{\upshape}{0.15em}{0.65em}{0.6em}%

\makeatletter
\def\maketitle{%
\par
\hrule height 0.75pt\vspace{1ex}
\par\noindent
\begin{minipage}{0.5\textwidth}
\scshape
purdue university $\cdot$ cs 59000-nmc \\
networks \textup{\textit{\footnotesize \&}} matrix computations
\end{minipage}
\begin{minipage}{0.5\textwidth}
\raggedleft
\MakeTextUppercase{\allcapsspacing{\@title}}\\[0.2ex]
\textit{\@author}\\[0.2ex]
\textit{\@date}
\end{minipage}
\par\vspace{1ex}
\hrule height 1pt
\vspace{2ex}
\par
}
\makeatother

\author{David F. Gleich}
\title{Lecture Notes}
% auto generate a title
\AtBeginDocument{\maketitle}



\begin{document} 
\hypertarget{homework_1_1}{}\section*{{Homework 1}}\label{homework_1_1}

\hypertarget{due_september_20th_2011_2}{}\subsection*{{Due September 20th, 2011}}\label{due_september_20th_2011_2}

\hypertarget{problem_1_norms_3}{}\subsection*{{Problem 1: Norms}}\label{problem_1_norms_3}

\emph{a)} Show that $f(\vx) = \normof{\mA \vx}_p$ is a vector-norm, where $\mA$ is a non-singular matrix.

\emph{b)} Show that $f(\vx) = \normof{\mA \vx}_p$ is not a vector-norm if $\mA$ is singular.

These norms will arise in our study of spectral graph theorem. In those cases, the matrix $\mA$ is usually the diagonal matrix of degrees for each node --{} commonly written $\mD$.

\hypertarget{problem_2_4}{}\subsection*{{Problem 2}}\label{problem_2_4}

There are a tremendous number of matrix norms that arise. An interesting class are called the \emph{orthgonally invariant norms}. Norms in this class satisfy:

\begin{displaymath}
\normof{\mA} = \normof{\mU \mA \mV}
\end{displaymath}
for \emph{square orthogonal matrices} $\mU$ and $\mV$. Recall that a square matrix is orthogonal when $\mU^T \mU = \mI$, i.e. $\mU^{-1} = \mU^T$.

\emph{a)} Show that $\normof{ \mA }_F$ is orthogonally invariant. (Hint: use the relationship between $\normof{ \mA }_F$ and $\text{trace}( \mA^T \mA )$.)

\emph{b)} Show that $\normof{ \mA }_2$ is orthogonally invariant. (Hint: first show that $\normof{ \mU \vx }_2 = \normof{\vx}_2$ using the relationshp between $\normof{ \vx }$ and $\vx^T \vx$.)

\hypertarget{problem_3_5}{}\subsection*{{Problem 3}}\label{problem_3_5}

In this problem, we'{}ll work through the answer to the challenge question on the introductory survey.

Let $\mA$ be the adjacency matrix of a simple, undirected graph.

\emph{a)} \textbf{An upper bound on the largest eigenvalue}\newline Show that $\lambda_{\max}(\mA)$ is at most, the maximum degree of the graph. Show that this bound is tight.

\emph{b)} \textbf{A lower bound on the largest eigenvalue} Show that $\lambda_{\max}(\mA)$ is at least, the square-root of the maximum degree of the graph. Show that this bound is tight. (Hint: try and find a lower-bound on the Rayleigh-Ritz characterization $\lambda_{\max} = \max \vx^T \mA \vx / \vx^T \vx$.)

\hypertarget{problem_4_6}{}\subsection*{{Problem 4}}\label{problem_4_6}

In this question, we'{}ll show how to use these tools to solve a problem that arose when Amy Langville and I were studying ranking algorithms.

\emph{a)} \textbf{the quiz from class} Let $\mA$ be an $n \times n$ matrix of all ones:

\begin{displaymath}
\mA = \bmat{ 1 & \cdots & 1 \\                \vdots & & \vdots \\
                1 & \cdots & 1 }.
\end{displaymath}
What are the eigenvalues of $\mA$? What are the eigenvectors for all non-zero eigenvalues? Given a vector $\vx$, how can you tell if it'{}s in the \emph{nullspace} (i.e. it'{}s eigenvector with eigenvalue 0) without looking at the matrix?

\emph{b)} \textbf{my problem with Amy} Amy and I were studying the $n+1 \times n+1$ matrix:

\begin{displaymath}
\mA = \bmat{ n & -1 & \cdots & -1 \\                -1 & \ddots & & \vdots \\
                \vdots & & \ddots & -1 \\
                -1 & \cdots & -1 & n }
\end{displaymath}
that arose when we were looking at ranking problems like we saw in \url{http://www.cs.purdue.edu/homes/dgleich/nmcomp/lectures/lecture-1-matlab.m} What we noticed was that Krylov methods to solve

\begin{displaymath}
\mA \vx = \vb
\end{displaymath}
worked incredibly fast.\newline Usually this happens when $\mA$ only has a few \emph{unique} eigenvalues. Show that this is indeed the case. What are the \emph{unique} eigenvalues of $\mA$?

\emph{c)} \textbf{solving the system} Once we realized that there were only a few unique eigenvalues and vectors, we wanted to determine if there was a closed form solution of:

\begin{displaymath}
\mA \vx = \vb.
\end{displaymath}
There is such a form. Find it. (By closed form, I mean, given $\vb$, there should be a simple expression for $\vx$.)

\hypertarget{problem_5_7}{}\subsection*{{Problem 5}}\label{problem_5_7}

In this question, you'{}ll implement codes to convert between triplet form of a sparse matrix and compressed sparse row.

\textbf{You may use any language you'{}d like.}

\emph{a)} Describe and implement a procedure to turn a set of triplet data this data into a one-index based set of arrays: {\colorbox[rgb]{1.00,0.93,1.00}{\tt pointers\char44\char32columns\char44\char32and\char32values}} for the compressed sparse form of the matrix. Use as little additional memory as possible. (Hint: it'{}s doable using \emph{no} extra memory.)

\begin{verbatim}function [pointers, columns, values] = sparse_compress(m, n, triplets)
% SPARSE_COMPRESS Convert from triplet form
%
% Given a m-by-n sparse matrix stored as triplets:
%   triplets(nzi,:) = (i,j,value)
% Output the  the compressed sparse row arrays for the sparse matrix.

% fill in the function\end{verbatim}
\emph{b)} Describe and implement a procedure to take in the one-indexed compressed sparse row form of a matrix: {\colorbox[rgb]{1.00,0.93,1.00}{\tt pointers\char44\char32columns\char44\char32and\char32values}} and the dimensions {\colorbox[rgb]{1.00,0.93,1.00}{\tt m\char44\char32n}} and output the compressed sparse row arrays for the transpose of the matrix:

\begin{verbatim}function [pointers_out, columns_out, values_out] = sparse_transpose(...
	m, n, pointers, columns, values)
% SPARSE_TRANSPOSE Compute the CSR form of a matrix transpose.
%
% 

% fill in the function\end{verbatim}
\hypertarget{problem_6_make_it_run_in_matlaboctavescipyetc_8}{}\subsection*{{Problem 6: Make it run in Matlab/Octave/Scipy/etc.}}\label{problem_6_make_it_run_in_matlaboctavescipyetc_8}

In this problem, you'{}ll just have to run three problems on matlab. The first one will be to use the Jacobi method to solve a linear system. The second will be to use a Krylov method to solve a linear system. The third will be to use ARPACK to compute eigenvalues on Matlab.

For this problem, you'{}ll need to use the `{}minnesota'{} road network.\newline It'{}s available on the website: \url{http://www.cs.purdue.edu/homes/dgleich/nmcomp/matlab/minnesota.mat} The file is in Matlab format. If you need another format, let me know.

\emph{a)} Use the {\colorbox[rgb]{1.00,0.93,1.00}{\tt gplot}} function in Matlab to draw a picture of the Minnesota road network.

\emph{b)} Check that the adjacency matrix A has only non-zero values of 1 and that it is symmetric. Fix any problems you encouter.

\emph{c)} We'{}ll do some work with this graph and the linear system described in class:

\begin{displaymath}
\mI - \gamma \mL
\end{displaymath}
where $\mL$ is the combinatorial Laplacian matrix.

\begin{verbatim} % In Matlab code
 L = diag(sum(A)) - A;
 S = speye(n) - gamma*L;\end{verbatim}
For the right-hand side, label all the points above latitude line 47 with 1, and all points below latitude line 44 with -1.

\begin{verbatim} % In Matlab code
 b = zeros(n,1);
 b(xy(:,2) > 47) = 1;
 b(xy(:,2) < 44) = -1;\end{verbatim}
Write a routine to solve the linear system using the Jacobi method on the compressed sparse row arrays. You should use your code from 5a to get these arrays by calling

\begin{verbatim} [src,dst,val] = find(S); 
 T = [src,dst,val];
 [pointers,columns,values] = sparse_compress(size(A,1), size(A,2), T);\end{verbatim}
Show the convergence, in the relative residual metric:

\begin{displaymath}
\normof{\vb - \mA \vx^{(k)}}/\normof{b}
\end{displaymath}
when {\colorbox[rgb]{1.00,0.93,1.00}{\tt gamma\char32\char61\char32\char49\char47\char55}} (Note that $\mA$ is the matrix in the linear system, not the adjacency matrix.)

Show what happens when {\colorbox[rgb]{1.00,0.93,1.00}{\tt gamma\char61\char49\char47\char53}}

\emph{d)} Try using Conjugate Gradient {\colorbox[rgb]{1.00,0.93,1.00}{\tt pcg}} and {\colorbox[rgb]{1.00,0.93,1.00}{\tt minres}} in Matlab on this same system with {\colorbox[rgb]{1.00,0.93,1.00}{\tt gamma\char61\char49\char47\char55}} and {\colorbox[rgb]{1.00,0.93,1.00}{\tt gamma\char61\char49\char47\char53}}. Show the convergence of the residuals.

\emph{e)} Use the {\colorbox[rgb]{1.00,0.93,1.00}{\tt eigs}} routine to find the 18 smallest eigenvalues of the Laplacian matrix $\mL$.


\end{document}
